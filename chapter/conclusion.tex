\chapter{Conclusion}

	There are several short term goals of the project now that uncorrelated errors can be efficiently simulated. 
	
	Randomized benchmarking is a method for characterizing the failure rate of imperfect quantum gates. While this is a well established area of research for many application of quantum gates, randomized benchmarking has never before been applied to the Toric Code. A local check code like the Toric Code is known to be robust against localized errors, so investigating this behaviour with randomized benchmarking is an useful area of research. Developing an implementation of randomized benchmarking for the Toric code, and investigating the behaviour of a local check code with faulty gates is an important future goal of the project.
	
	Information about correlation distances can be garnered from syndrome measurements. Optimizing the decoder performance using information gained from syndrome measurements about correlated error models should increase the success rate of the code. Improving the success rates of ECCs is much sought after, and is a priority of the project moving forward. 
	
	\section{Future work}
	
	Future work may include creating a bias for walks to travel in one direction. This would be akin to creating markov chain montecarlo method.
	
	 Other possible work could be having walks avoid each other? 